\chapter{Serial}
\section{Serial.begin(rate)}

Abre el puerto serie y fija la velocidad en baudios para la transmisión de datos en serie. El valor típico de velocidad para comunicarse con el ordenador es 9600, aunque otras velocidades pueden ser soportadas.
\begin{lstlisting}
void setup()

{
Serial.begin(9600);    // abre el Puerto serie
}    // configurando la velocidad en 9600 bps
\end{lstlisting}
\textbf{Nota}: Cuando se utiliza la comunicación serie los pines digitales 0 (RX) y 1 (TX) no pueden utilizarse para otros propósitos.
\section{Serial.println(data)}

Imprime los datos en el puerto serie, seguido por un retorno de carro y salto de línea. Este comando toma la misma forma que Serial.print (), pero es más fácil para la lectura de los datos en el Monitor Serie del software.
\begin{lstlisting}
Serial.println(analogValue);    // envia el valor 'analogValue' al puerto
\end{lstlisting}
\textbf{Nota}: Para obtener más información sobre las distintas posibilidades de Serial.println() y Serial.print() puede consultarse el sitio web de Arduino.
\newpage{}
El siguiente ejemplo toma de una lectura analógica del pin 0 y envía estos datos al ordenador cada segundo.
\begin{lstlisting}
void setup()
{
Serial.begin(9600);    // configura el puerto serie a 9600bps
}

void loop()
{
Serial.println(analogRead(0)); // envia valor analogico
delay(1000);     // espera 1 segundo
}
\end{lstlisting}

\section{Serial.println(data, data type)}

Vuelca o envía un número o una cadena de caracteres al puerto serie, seguido de un caracter de retorno de carro 'CR' (ASCII 13, or '$\backslash$r')y un caracter de salto de línea 'LF'(ASCII 10, or '$\backslash$n'). Toma la misma forma que el comando Serial.print()
\begin{itemize}
\item Serial.println(b) vuelca o envía el valor de b como un número decimal en caracteres ASCII seguido de 'CR' y 'LF'.
\item Serial.println(b, DEC) vuelca o envía el valor de b como un número decimal en caracteres ASCII seguido de 'CR' y 'LF'.
\item Serial.println(b, HEX) vuelca o envía el valor de b como un número hexdecimal en caracteres ASCII seguido de 'CR' y 'LF'.
\item Serial.println(b, OCT) vuelca o envía el valor de b como un número octal en caracteres ASCII seguido de 'CR' y 'LF'.
\item Serial.println(b, BIN) vuelca o envía el valor de b como un número binario en caracteres ASCII seguido de 'CR' y 'LF'.
\item Serial.print(b, BYTE) vuelca o envía el valor de b como un byteseguido de 'CR' y 'LF'.
\item Serial.println(str) vuelca o envía la cadena de caracteres como una cadena ASCII seguido de 'CR' y 'LF'.
\item Serial.println() sólo vuelca o envía 'CR' y 'LF'.
\end{itemize}
\section{Serial.print(data, data type)}

Vuelca o envía un número o una cadena de carateres, al puerto serie. Dicho comando puede tomar diferentes formas, dependiendo de los parámetros que utilicemos para definir el formato de volcado de los números.
\subsection{Parámetros}
\begin{itemize}
\item data: el número o la cadena de caracteres a volcar o enviar.
\item data type: determina el formato de salida de los valores numéricos (decimal, octal, binario, etc...) DEC, OCT, BIN, HEX, BYTE.
\end{itemize}
\subsection{Ejemplos}
\textbf{Serial.print(b)}
Vuelca o envía el valor de b como un número decimal en caracteres ASCII.
\begin{lstlisting}
int b = 79; Serial.print(b); // envia "79".
\end{lstlisting}
\textbf{Serial.print(b, DEC)}
Vuelca o envía el valor de b como un número decimal en caracteres ASCII.
\begin{lstlisting}
int b = 79;
Serial.print(b, DEC); // envia "79".
\end{lstlisting}
\textbf{Serial.print(b, HEX)}
Vuelca o envía el valor de b como un número hexdecimal en caracteres ASCII.
\begin{lstlisting}
int b = 79;
Serial.print(b, HEX); // envia "4F".
\end{lstlisting}
\textbf{Serial.print(b, OCT)}
Vuelca o envía el valor de b como un número octal en caracteres ASCII.
\begin{lstlisting}
int b = 79;
Serial.print(b, OCT); // envia "117".
\end{lstlisting}
\newpage{}
\textbf{Serial.print(b, BIN)}
Vuelca o envía el valor de b como un número binario en caracteres ASCII.
\begin{lstlisting}
int b = 79;
Serial.print(b, BIN); // envia "1001111".
\end{lstlisting}
\textbf{Serial.print(b, BYTE)}
Vuelca o envía el valor de b como un byte.
\begin{lstlisting}
int b = 79;

Serial.print(b, BYTE); // Devuelve el caracter 'O', el cual representa el caracter ASCII del valor 79. (Ver tabla ASCII ).
\end{lstlisting}
\textbf{Serial.print(str)}
Vuelca o envía la cadena de caracteres como una cadena ASCII.
\begin{lstlisting}
Serial.print("Hello World!"); // envia "Hello World!".
\end{lstlisting}

\section{Serial.avaible()}
\begin{lstlisting}
int Serial.available()
\end{lstlisting}
Devuelve un entero con el número de bytes (carácteres) disponibles para leer desde el buffer serie, ó 0 si no hay ninguno. Si hay algún dato disponible, SerialAvailable() será mayor que 0. El buffer serie puede almacenar como máximo 128 bytes.\\
Ejemplo:
\begin{lstlisting}
int incomingByte = 0; // almacena el dato serie
void setup() {
Serial.begin(9600); // abre el puerto serie, y le asigna la velocidad de 9600 bps
}
void loop() {
// envia datos solo si los recibe:
if (Serial.available() > 0) {
// lee el byte de entrada:
incomingByte = Serial.read();
//lo vuelca a pantalla
Serial.print("He recibido: "); Serial.println(incomingByte, DEC);
}
}
\end{lstlisting}
\section{Serial.Read()}
\begin{lstlisting}
int Serial.Read()
\end{lstlisting}
Lee o captura un byte (carácter) desde el puerto serie. Devuelve :El siguiente byte (carácter) desde el puerto serie, ó -1 si no hay ninguno.
Ejemplo
\begin{lstlisting}
int incomingByte = 0; // almacenar el dato serie
void setup() {
Serial.begin(9600); // abre el puerto serie,y le asigna la velocidad de 9600 bps
}
void loop() {
// envia datos solo si los recibe:
if (Serial.available() > 0) {
// lee el byte de entrada:
incomingByte = Serial.read();
//lo vuelca a pantalla
Serial.print("He recibido: "); Serial.println(incomingByte, DEC);
}
}
\end{lstlisting}