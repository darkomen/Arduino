\chapter{Estructuras de control}
\section{if (si condicional)}

if es un estamento que se utiliza para probar si una determinada condición se ha alcanzado, como por ejemplo averiguar si un valor analógico está por encima de un cierto número y ejecutar una serie de declaraciones (operaciones) que se escriben dentro de llaves, si es verdad. Si es falso (la condición no se cumple) el programa salta y no ejecuta las operaciones que están dentro de las llaves. El formato para if es el siguiente:
\begin{lstlisting}
if (unaVariable ?? valor)
{
ejecutaInstrucciones;
}
\end{lstlisting}
En el ejemplo anterior se compara una variable con un valor, el cual puede ser una variable o constante. Si la comparación, o la condición entre paréntesis se cumple (es cierta), las declaraciones dentro de los corchetes se ejecutan. Si no es así, el programa salta sobre ellas y sigue.\\\\
\textbf{Nota}: Tenga en cuenta el uso especial del símbolo '=', poner dentro de if (x = 10), podría parecer que es valido pero sin embargo no lo es ya que esa expresión asigna el valor 10 a la variable x, por eso dentro de la estructura if se utilizaría x == 10 que en este caso lo que hace el programa es comprobar si el valor de x es 10. Ambas cosas son distintas por lo tanto dentro de las estructuras if, cuando se pregunte por un valor se debe poner el signo doble de igual ==.

\section{if… else (si .. sino ..)}

if… else viene a ser un estructura que se ejecuta en respuesta a la idea "si esto no se cumple haz esto otro". Por ejemplo, si se desea probar una entrada digital, y hacer una cosa si la entrada fue alto o hacer otra cosa si la entrada es baja, usted escribiría que de esta manera:
\begin{lstlisting}
if (inputPin == HIGH)
{
instruccionesA;    
}
else   
{
instruccionesB;
}
\end{lstlisting}
else puede ir precedido de otra condición de manera que se pueden establecer varias estructuras condicionales de tipo unas dentro de las otras (anidamiento) de forma que sean mutuamente excluyentes pudiéndose ejecutar a la vez. Es incluso posible tener un número ilimitado de estos condicionales. Recuerde sin embargo qué sólo un conjunto de declaraciones se llevará a cabo dependiendo de la condición probada:
\begin{lstlisting}
if (inputPin < 500)
  {
  instruccionesA;
  }
else if (inputPin >= 1000)
  {
  instruccionesB;
  {
else   
  {
  instruccionesC;
  }
\end{lstlisting}
\textbf{Nota}: Un estamento de tipo if prueba simplemente si la condición dentro del paréntesis es verdadera o falsa. Esta declaración puede ser cualquier declaración válida. En el anterior ejemplo, si cambiamos y ponemos (inputPin == HIGH). En este caso, el estamento if sólo chequearía si la entrada especificado esta en nivel alto (HIGH), ó +5v.
\section{For}

La declaración for se usa para repetir un bloque de sentencias encerradas entre llaves un número determinado de veces. Cada vez que se ejecutan las instrucciones del bucle se vuelve a testear la condición. La declaración for tiene tres partes separadas por ';' , veamos el ejemplo de su sintaxis:
\begin{lstlisting}
for (inicializacion; condicion; expresion)
{
Instrucciones;
}
\end{lstlisting}
La inicialización de una variable local se produce una sola vez y la condición se testea cada vez que se termina la ejecución de las instrucciones dentro del bucle. Si la condición sigue cumpliéndose, las instrucciones del bucle se vuelven a ejecutar. Cuando la condición no se cumple, el bucle termina.\\\\
El siguiente ejemplo inicia el entero i en el 0, y la condición es probar que el valor es inferior a 20 y si es cierto i se incrementa en 1 y se vuelven a ejecutar las instrucciones que hay dentro de las llaves:
\begin{lstlisting}
for (int i=0; i<20; i++)    // declara i y prueba si es
{                // menor que 20, incrementa i.
digitalWrite(13, HIGH);    // enciende el pin 13
delay(1000);             // espera un seg.
digitalWrite(13, LOW);     // apaga el pin 13
delay(1000);             // espera un seg.
}
\end{lstlisting}
\textbf{Nota}: El bucle en el lenguaje C es mucho más flexible que otros bucles encontrados en algunos otros lenguajes de programación, incluyendo BASIC. Cualquiera de los tres elementos de cabecera puede omitirse, aunque el punto y coma es obligatorio. También las declaraciones de inicialización, condición y expresión puede ser cualquier estamento válido en lenguaje C sin relación con las variables declaradas. Estos tipos de estados son extraños pero permiten crear soluciones a algunos problemas de programación específicos.
\section{while}

Un bucle del tipo while es un bucle de ejecución continua mientras se cumpla la expresión colocada entre paréntesis en la cabecera del bucle. La variable de prueba tendrá que cambiar para salir del bucle. La situación podrá cambiar a expensas de una expresión dentro el código del bucle o también por el cambio de un valor en una entrada de un sensor.
\begin{lstlisting}
while (unaVariable ?? valor)
{
ejecutarSentencias;
}
\end{lstlisting}
El siguiente ejemplo testea si la variable unaVariable es inferior a 200 y si es verdad, ejecuta las declaraciones dentro de los corchetes y continuará ejecutando el bucle hasta que unaVariable no sea inferior a 200.
\begin{lstlisting}
while (unaVariable < 200)    // testea si es menor que 200
{
instrucciones;        // ejecuta las instrucciones
                      // entre llaves
unaVariable++;        // incrementa la variable en 1
}
\end{lstlisting}
\section{do… while}

El bucle do… while funciona de la misma manera que el bucle while, con la salvedad de que la condición se prueba al final del bucle, por lo que el bucle siempre se ejecutará al menos una vez.
\begin{lstlisting}
do
{
Instrucciones;
} while (unaVariable ?? valor);
\end{lstlisting}
El siguiente ejemplo asigna el valor leído leeSensor() a la variable x, espera 50 milisegundos y luego continua mientras que el valor de la x sea inferior a 100.
\begin{lstlisting}
do
{
x = leeSensor();
delay(50);
} while (x < 100);
\end{lstlisting}